\chapter{Documentación Semi-Formal de la Seguridad del Web Browser}
\label{chap:chap4}

\section{Sistemas Seguros}
\label{chap:chap4.1}
En la Literatura existen variadas formas de construir sistemas seguros 
[mencionar al menos 3 y explicarlos en 3 parrafos].
-CLASP
-Microsoft

El enfoque en el cuál se basará este estudio es aquel que imparte Fernandez \cite{fernandez2013security}, que sostiene que para construir un sistema seguro es necesario realizarlo de manera sistemática de tal manera que la seguridad sea parte del integral de cada una de las etapas del Desarrollo de Software - de inicio a fin. El enfoque que propone es ingenieril y por tanto es aplicable incluso para sistemas \textit{legacy}, donde es posible hacer ingeniería inversa para comprobar si existen o no los requerimiento de seguridad implementados, de manera que permite generar un estudio con la intención de comparar y mejorar nuevos sistemas. En su libro \cite{ref1} presenta una completa metodología para construir sistemas seguros a partir de patrones de diseño, a los cuales nombra como \textbf{Security Patterns}.

Como parte de la metodología propuesta, se plantea que para diseñar primero se deben entender las posibles amenazas a las que está expuesto el sistema. La identificación de Amenazas [Bra08a] [Fer06c] es la primera tarea que presenta la metodología, que considera las actividades en cada caso de uso del sistema.