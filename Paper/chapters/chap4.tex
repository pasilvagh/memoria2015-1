\chapter{Documentación Semi-Formal de la Seguridad del Web Browser}
\label{chap:chap4}

\section{Secure Software Development y Secure Software Design}
\label{chap:chap4.1}

La filosofía detrás de \textit{Secure Software Developmet} es que detrás de cada etapa de desarrollo del software, se tengan en cuenta que los prinicipios de Seguridad: Confidencialidad, Integridad, Disponibilidad, Auditoría. Para cumplir este cometido es que se deben llegar a políticas y reglas que aseguren la Seguridad como una propiedad sistémica.

Varias comunidades tienen diferentes enfoques y técnicas de cómo asegurar la Seguridad en los sistemas, muchas pueden incluso tener similitudes y hasta trabajar juntas. En este trabajo, el enfoque tomado es aquél que busca entregar la propiedad de seguridad a través del entendimiento de un sistema a un alto nivel, identificando las amenazas durante la elicitación de requerimientos, de manera que se pueda extraer las posibles amenazas que podrían existir y utilizando elementos de diseño para hacer cumplir los principios de seguridad necesarios por el sistema; este enfoque es el que se dedica la comunidad de \textit{Secure Software Design}. 


Fernandez \cite{fernandez2013security} sostiene que para construir un sistema seguro es necesario realizarlo de manera sistemática de tal manera que la seguridad sea parte del integral de cada una de las etapas del Desarrollo de Software - de inicio a fin. El enfoque que propone es ingenieril y por tanto es aplicable incluso para sistemas \textit{legacy}, donde es posible hacer ingeniería inversa para comprobar si existen o no los requerimiento de seguridad implementados, de manera que permite generar un estudio con la intención de comparar y mejorar nuevos sistemas. En su libro \cite{ref1} presenta una completa metodología para construir sistemas seguros a partir de patrones de diseño, a los cuales nombra como \textbf{Security Patterns}.

Como parte de la metodología propuesta, se plantea que para diseñar primero se deben entender las posibles amenazas a las que está expuesto el sistema. La identificación de Amenazas [Bra08a] [Fer06c] es la primera tarea que presenta la metodología, que considera las actividades en cada caso de uso del sistema.


\section{Arquitectura de Referencia - \textit{Reference Architecture}}
%usar: grosskurth2005, webpag3 (IE), reis2009isolating

Una arquitectura de Referencia, de acuerdo a la \textit{Open Security Architecture} o OSA\cite{openSecArch}, es considerado un elemento que describe un \textbf{estado de ser} y debe representar aceptadas buenas practicas. 
%Buscar si existe una definición para el caso de Secure Software Desing Community

En este trabajo lo que se pretende hacer con la Arquitectura de Referencia, es dar a entender los componentes y elementos que la mayoría de los Web Browser tienen. Se sabe que el Browser es un pieza de Software que ha sufrido varios cambios desde 1990, por lo tanto entre los desarrolladores de ésta herramienta ya existen conveniones de qué elementos funcionan mejor. Por consiguiente, no es de extrañar que diferentes browsers estén construidos de formas muy similares, incluso existan ciertos \textbf{patrones} que pueden explayarse de buena manera mediante una Arquitectura de Referencia, que manifeste los componentes, mecanismos de comunicación, funcionamientos, etc.

\section{Security Patterns o Patrones de Seguridad}

\section{Misuse Patterns o Patrones de Mal Uso}
Para diseñar sistemas seguros, se es necesario identificar las posibles amenazas que un sistema puede sufrir. Papers como \cite{fernandez2013security, fernandez2006defining, fernandez2007attack, braz2008eliciting} describen el desarrollo de una metodología completa para encontrar amenazas, a través del análisis de actividaddes de los casos de uso del sistema buscando como podría un atacante interno o externo socavar las bases de esas actividades. Es importante no confundir \textit{Attack Patterns} con \textit{Misuse Pattern}, pués claramente en \cite{ModMisusePatt, fernandez2013security} dejan explícito que un \textit{Attack Patttern} es una acción que lleva a un mal uso o \textit{misuse}, o acciones \textbf{específicas} que toman ventaja de las vulnerabilidades de un sistema, como por ejemplo un \textit{buffer overflow}. A partir de los trabajos \cite{fernandez2007attack, yoshioka2006development, yoshioka2007integration}  se hace la unión de los conceptos de \textit{Attack Patttern} para dar forma a la definición de \textit{Misuse Pattern} \cite{ModMisusePatt, pelaez2009misuse, fernandez2010worm, hashizume2011misuse, munoz2011misuse, fernandez2012misuse, alkazimi2014, encinamisuse}. Esta nueva definición indica entonces indica que:
\begin{center}
	Un patrón de mal uso o \textit{Misuse Pattern} describe, desde el punto de vista del atacante, qué tipo de ataque es realizadom ()
\end{center}
The misuse pattern describes, from the point of view of the attacker, how a type of at-
tack is performed (what units it uses and how), analyzes ways of stopping the attack by
enumerating possible security patterns that can be applied for this purpose, and describes
how to trace the attack once it has happened by appropriate collection and observation
of forensic data. It also describes precisely the context in which the attack may occur



Sin embargo, cuando un sistema ya está diseñadom y construido, como es el caso del Web Browser, lo que va a importar es saber \textbf{cómo} los componentes del sistema, pueden ser usados por el atacante para alcanzar sus objetivos. Un \textit{Misuse Pattern} o \textbf{Patrón de Mal Uso} describe, desde el punto de vista del atacante, cómo un tipo de ataque es realizado, indicando \textbf{qué} componentes usa y \textbf{cómo}. Además analiza las formas de detener el ataque a través de un listado de posibles \textit{Security Patterns} o \textbf{Patrones de Seguridad} que pueden ser aplicados para esa situación, y describe cómo poder seguir el rastro de un ataque una vez que ha sido realizado con éxito en el sistema, a través de data forense. Además describe un contexto en dónde puede ocurrir el ataque.

