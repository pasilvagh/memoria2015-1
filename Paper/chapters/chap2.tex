\chapter{Estado del Arte: El Web Browser}
\label{chap:chap2}

\section{Same Origin Policy}
\label{chap:SOP}

Este importante concepto nace a partir del Modelo de Seguridad detrás de una Aplicación Web, al mismo tiempo que es el mecanismo más básico que el Browser tiene para protegerse de las amenazas que aparecen en el día a día, haciendo un poco más complicado el trabajo de realizar un \textit{exploit}. \textbf{Same Origin Policy} o \textbf{SOP} define lo que es un \textbf{Origen}, compuesto por el \textit{esquema}, el \textit{host/dominio} y \textit{puerto} de la URL. Esta política permite que un Web Browser aisle los distintos recursos obtenidos por las páginas web y que solo permita la ejecución de \textit{Script} que pertenezcan a un misno \textbf{Origen}. 

\textbf{SOP} puede distinguir entre la información que envía y recibe el Web Browser, y solo se aplicará la política a los elementos externos que se soliciten dentro de una página web (recepción de la información). Esta imposibilidad de recibir información de un \textbf{Origen} diferente al del recurso actual, permite disminuir la superficie de ataque y la posibilidad de explotar alguna vulnerabilidad en el sistema donde reside el Browser. Sin embargo, \textbf{SOP} no pone ninguna restricción sobre la información que el usuario puede enviar hacia otros. 


%Ver libro de Browser hacker handbook

\section{HTML:}
\label{chap:HTML}

\section{Webworkers}

\section{}

\section{HTTP}
\label{chap:HTTP}

\subsection{Comunicación en HTTP}
\label{chap:comunHTTP}

\subsubsection{postMessage}
\label{chap:postmessage}

\subsubsection{XMLHttpRequest}
\label{XMLHR}

\subsubsection{WebSockets}
\label{WebSockets}

