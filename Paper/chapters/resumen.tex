% Resumen a grandes rasgos de lo existente (resumen del estado del arte y resumen del trabajo realizado)

\section*{Resumen}
\label{chap:resumen}

El Web Browser es una de las aplicaciones más usadas - \textit{killer app} - y también una de las primeras en aparecer en cuanto se creó el Internet (Década de los 90). Por lo mismo, su nivel de madurez con respecto a otros desarrollos es significativo y permite asegurar ciertos niveles de confianza cuando otros usan un Web Browser como cliente para sus Sistemas. 

En la actualidad, muchos Desarrollos de Software usan ésta técnología dado que grandes compañías como Google, Microsoft, Mozilla, entre la más importantes, dedican la mayoría de su tiempo en asegurar la calidad de sus Navegadores Web, lo que se traduce en un gasto menor de esfuerzo en el Desarrollo. Sin embargo, dado que la misma Internet es considerada como una red no confiable, estos sistemas que usan al navegador como Cliente Web, introducen nuevas amenazas de seguridad que deben ser tomadas en cuenta en el Desarrollo del sistema.

Esta Memoria incursionará en el ámbito de la seguridad del Web Browser, con el objetivo de obtener documentos formales que servirán como herramientas a personas que Desarrollen Software y hagan un fuerte uso del Navegador para las actividades del sistema desarrollado.



\section*{Abstract}
\label{chap:abstract}


