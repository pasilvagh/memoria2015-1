
%Intro CHECKED!!


\chapter{Introducción}
\label{chap1:intro}

\section{Contexto General}
\label{chap1:CG}

Entre 1989 y 1990, Tim Berners-Lee acuño el concepto de \textit{World Wide Web} y con ésto realizó la construcción del primer \textit{Web Server}, \textit{Web Browser} y las primeras páginas \textit{Web}. Mucho antes que aparecieran los grandes sistemas que ahora conocemos, el \textit{Web Browser} era la herramienta usada por todos para navegar, las ya viejas páginas estáticas, y permitir un sin número de acciones a aquellos que necesitaban conectarse al \textit{Internet}. En la actualidad el \textit{browser} sigue siendo la herramienta predilecta por todos, desde comprar tickets para una película, realizar reuniones por videoconferencia y muchas otras tareas que invitan a nuevas formas de interactuar y comunicar.


En el último tiempo el mercado de los \textit{Web Browser} ha crecido bastante , esto es debido principalmente a la robustez que éstos poseen y a la cantidad de años que llevan desarrollandose en la industria del Desarrollo de Software. Los navegadores más conocidos son: Google Chrome/Chromium, Firefox, Internet Explorer, Opera, and Safari. 

La \textit{Web 2.0}, que se inició con el uso intensivo de tecnologías como AJAX, ha permitido una nueva simbiosis entre el usuario y el Web Browser. Haciendo de éste último, una herramienta casi indispensable para todo tipo de tareas computacionales, tal que su existencia a penetrado completamente en las labores diarias de todos nosotros. En este mismo instante, la Web a evolucionado nuevamente obteniendo un nuevo nombre: \textit{Web 3.0}, donde se realiza el uso de inteligencia artificial y sistemas de recomendación para generar nuevos tipos de contenido media para el usuario.


\section{El Problema: Desarrollo de Software y Seguridad}
\label{chap1:SD_SS}

Ningún Desarrollo de Software es igual al anterior. Cada vez que un nuevo Proyecto surge es necesario ver qué tipo de Proceso es el que se usará, qué personas serán parte del grupo de trabajo, qué condiciones económicas estará expuesto, qué tipo de \textit{Stakeholder} están pendientes de que el Proyecto salga exitoso y un sin numero de variables, no menos importantes a considerar. Por lo tanto dependiendo de lo anterior, los sistemas podrían llegar a ser simples o muy complejos y por consiguiente, se hace necesario tener ciertas metodologías que aseguren que se cumplan con todos los Requerimientos Funcionales como No-Funcionales del Sistema a construir. El problema de la mayoría del Software construído es que contiene numerosos \textbf{defectos} y \textbf{errores}, generando así \textbf{vulnerabilidades} que son encontradas y explotadas por los atacantes, de tal manera que generan el compromiso del sistema completo \cite{goertzel2007software}.

El fenómeno en la literatura llamado \textit{Zero-day attack} se refiere al momento clave donde un atacante explota una vulnerabilidad - hasta ese momento desconocida - de algún sistema (importante o no), y que si no es parchado lo antes posible puede comprometer no solo a sistemas, si no también a los usuarios que hacen uso de éste. Sin embargo muchas veces ocurre que aunque se corrijan estos nuevos ataques, no todos los sistemas que podrían llegar a necesitar del mismo parche para protegerse del ataque, realizan la actualización y su adecuada configuración dpara así protegerse de una posible amenaza que explote la vulnerabilidad recientemente encontrada.

Si bien el \textbf{Zero-day Attack} es un evento que podría no ocurrir tan repetidamente, dado que se produce por el largo estudio llevado por el atacante, sobre el sistema a vulnerar, existen otras formas de comprometer a un sistema. Muchas veces al desarrollar sistemas, se prefiere utilizar API's\footnote{Application Programming Interface} de otros sistemas para poder incluir funcionalidades ya implementadas, fomentando así el Reuso de piezas de Software. Si bien lo anterior es una buena práctica, si el sistema no cuenta con las medidas de seguridad necesarias, estas piezas podrían ser causa de amenazas de seguridad que terminarían por corromper el sistema y en consecuencia podría causar una pérdida monetaria a los \textit{Stakeholders}. 

Es por lo expuesto anteriormente, que lidiar con las preocupaciones de seguridad es un factor que puede ser clave para el Desarrollo de Software: Una vez que el sistema este en \textit{Deployment}, las consecuencias de no tener en cuenta la Seguridad desde el inicio del Desarrollo de Software pueden ser muy costosas \cite{cert}, incluso pudiendo afectar la Confidencialidad, Integridad y Disponibilidad de los datos de quienes lo usan \cite{interCoursera}. Por esto mismo, es imperante que sean entendidos los Requerimientos de Seguridad del Sistema a construir desde el inicio de su construcción y que todos los involucrados también los entiendan perfectamente. La literatura que habla de la construcción de \textit{Secure Software} o Software Seguro, indica que los practicantes del Desarrollo de Software deben entender, en gran medida, los problemas de seguridad que podrían llegar a ocurrir en sus sistemas. No basta con saber como unir las piezas, no basta con que cada pieza de por si sea segura, si los componentes del sistema no actuan de forma coordinada, probablemente éste no será seguro \cite{fernandez2013security}, dado que la seguridad es una Propiedad Sistémica que necesita ser vista de manera holística.  

%Un sistema seguro o \textbf{Secure Software}, no tendrá en lo posible, vulnerabilidades que puedan ser explotadas. Sin embargo, hay que tener en cuenta que dado que el Desarrollo de Software - incluso los sistemas seguros - dependen de personas, procesos, y tecnologías, será imposible tener un \(100\%\) de confiabilidad en la Seguridad implementada, pues los ``fallas humanas'' o ``descuidos humanos'' siempre existirán. Si el sistema seguro no pudiera llegar a resistir un ataque, lo primero que debe intentar es aislarse del resto y degradar con gracia, de manera que no afecte el rendimiento del sistema. Finalmente, si un ataque tuvo a lugar lo importante es que el equipo detrás del sistema no se quede inmovil, luego de recuperarse del compromiso es necesario tener inmediatamente un parche que solucione la vulnerabilidad existente. Más aún, si esta vulnerabilidad se encontrara en un componente externo o \textit{Third Party code}, el sistema que llegara a hacer uso de esa pieza de Software tiene que tener medidas para que no afecte la totalidad del sistema.

\section{Motivación: ¿Por qué estudiar el Browser?}
\label{chap1:motiv}
%En el pasado era bien visto que cada Desarrollo de Software, pudiera levantar todo su sistema sin tener que depender de externos para funcionar. Este pensamiento, sin embargo, ha cambiado mucho en el último tiempo tanto debido a factores sociales, económicos, técnicos y otros. 
Con la aparición de la \textit{Web 2.0 y 3.0}, con el uso de \textit{AJAX}, inteligencia artificial y sistemas de recomendación, permitieron nuevas formas de interacción entre usuarios y sistemas, lo que causó que el browser fuera usado extensivamente en los nuevos Desarrollos de Software dado que:
\begin{itemize}
	\item Permite disminuir los costos de construir un programa Cliente (desde cero) para el usuario del sistema.
	\item Actualmente la Seguridad implementada en los \textit{Web Browser} es bastante buena, dado que existen grandes compañias que se aseguran de ello (Google, Microsft, Mozilla entre las más conocidas). 
	\item El \textit{browser} es una herramienta indispensable. La mayoría de los sistemas que lo usan en la vida cotidiana son de tipo: \textit{online banking}, declaración de impuestos, promoción de empresas o tiendas, compras, y mucho más.
\end{itemize}


Sin embargo los sistemas que dependen del uso del \textit{Browser}, deben de tener en cuenta las posibles anemazas de seguridad a las que se enfrentarán por el solo hecho de usarlo. Para un proyecto de gran envergadura, sería un error no tener en consideración los posibles peligros que trae el uso del \textit{Browser}, y es el deber de todo integrante del equipo de Desarrollo tener el conocimiento de la seguridad del Cliente. Si bien existen metodologías creadas con la sola intención de entregar las herramientas para crear un software seguro, mucho antes de eso es necesario que los participantes comprendan los conceptos de seguridad a los que el sistema a construir se verá afectado. 

%(Nota: preguntar en otras universidades). 
En \cite{goertzel2007software} hace referencia a un hecho que personalmente la autora considera que también es una realidad en Chile, ésta es: La falta de cursos o educación en temas de Seguridad a los estudiantes de Ingeniería de Software. De tal manera que éstos estudiantes no aprenden acerca de Principios de Diseño en Seguridad, ni técnicas que permitan una segura implementación de código, a menos que lo necesiten en algún momento. Más aún, la falta de este tipo de conocimiento puede hacer creer que la seguridad es un requerimiento que puede o no ser tomado en cuenta al comienzo del Desarrollo. En este trabajo el enfoque es otro, la seguridad es una propiedad sistémica que debe ser tomada en cuenta desde el inicio del sistema \cite{goertzel2007software, braz2008eliciting, fernandez2013security}.

Este trabajo tiene una motivación principal. Ésta es ayudar a quién lo necesite con el conocimiento necesario para entender el funcionamiento y construcción del Cliente - el Web Browser-, los beneficios detrás de la Seguridad implementada en el Browser y de los peligros existentes de los que nos protegen. De esta manera se espera que alguien que lea este trabajo, tanto Estudiantes como Desarrolladores de Softwares, obtengan el conocimiento necesario al momento de trabajar junto con el Navegador Web al realizar un Desarrollo de Software que dependa de éste.



\section{Contribuciones}
\label{chap1:contr}

El Objetivo General de esta Memoria es generar un cuerpo organizado de información sobre el Web Browser y su Seguridad, de tal manera que se pueda sistematizar, organizar y clasificar el conocimiento adquirido en un documento, con formato semi-formal, tanto para Profesionales como Estudiantes del área Informática que estén insertos en el área de Desarrollo de Software. 

Este trabajo busca cumplir con los siguientes Objetivos Específicos:

\begin{itemize}
	\item Comprender los conceptos relacionados al navegador web, sus componentes, interacciones o formas de comunicación, amenazas y ataques a los que puede estar sometido, como los también los mecanismos de defensa. Esto se realizará a través de un Estado del Arte sobre el Browser.
	\item Construir una Arquitectura de Referencia del Web Browser e iniciar un pequeño catálogo de Patrones de Mal Uso o de Uso Indebido. Esto permitirá condensar el conocimiento obtenido en el punto anterior a través de documentos semi-formales, lo que permitirá generar una guía para comunicar los conceptos relevantes que pudieran afectar la relación existente entre alguna entidad y el navegador.
	\item Clasificar los ataques y mecanismos de defensa (mitigación) de los navegadores Web. %(REVISAR)
	\item Profundizar el conocimiento en ataques relacionados con métodos de Ingeniería Social.
	
\end{itemize} 

Particularmente se ha escogido como metodología base la dada por \cite{fernandez2013security}. En ésta se trata de llegar a una Arquitectura de Referencia del sistema que permite asegurar la construcción de sistemas seguros, mientras que los Patrones de Mal Uso permiten enseñar y comunicar las posibles formas en que tal sistema puede ser usado inapropiadamente. Para realizar lo anterior, esta Memoria propone el uso de patrones para abstraer las architecturas de software de los principales Navegadores Web del mercado: Internet Explore, Google Chrome/Chromium y Firefox.

En este trabajo se presentan X patrones de architectura, que representan en conjunto nuestra Arquitectura de Referencia. Además se presentan 2 Patrones de Uso Indebido, que usarán la Architectura de Referencia contruida para mostrar los componentes y mensajes que una amenaza puede realizar, con tal de lograr un ataque en el Browser. Estos patrones serán presentados usando el template POSA \cite{buschman1996system} y UML, para así modelar las interacciones entre los diversos componentes de la arquitectura.

\section{Metodología}
\label{chap1:Met}
Este trabajo se realizará de la siguiente forma:
\begin{enumerate}
	\item Contruir un Estado del Arte sobre el Browser, especialmente en los temas sobre la seguridad de éste.
	\item Identificar los conceptos, actores, componentes, interacciones y funciones, en relación al tema principal.
	\item Construir patrones de arquitectura que definan los componentes y responsabilidades, con el objetivo final de ser unidos en una Arquitectura de Referencia.
	\item Contruir patrones de Mal Uso/Uso Indebido por medio del punbto anterior.
\end{enumerate}

\section{Estructura del Documento}
\label{chap1:estruct}

El presente documento trata del trabajo de Memoria que se divide en las siguientes partes:

\begin{itemize}
	\item En el capítulo \ref{chap:chap2}... % se presentará la información necesaria para poder construir un completo Estado del Arte sobre el \textbf{Browser}, de tal manera de entender su funcionamientos, componentes e interacciones que realiza con otras entidades. 
	\item Luego de tener un extenso conocimiento de lo que actualmente es conocido como \textbf{Web Browser}, el capítulo \ref{chap:chap3}
\end{itemize}












